%!TEX TS-program = xelatex 
%!TEX TS-options = -synctex=1 -output-driver="xdvipdfmx -q -E"
%!TEX encoding = UTF-8 Unicode
%
%  priscian
%
%  Created by Mark Eli Kalderon on 2016-07-26.
%  Copyright (c) 2016. All rights reserved.
%

\documentclass[12pt]{article} 

% Definitions
\newcommand\mykeywords{Priscian, perception}
\newcommand\myauthor{Mark Eli Kalderon}
\newcommand\mytitle{Priscian on Perception}

% Packages
\usepackage{geometry} \geometry{a4paper} 
\usepackage{url}
% \usepackage{txfonts}
\usepackage{color}
\usepackage{enumerate}
\definecolor{gray}{rgb}{0.459,0.438,0.471}
% \usepackage{setspace}
% \doublespace % Uncomment for doublespacing if necessary
% \usepackage{epigraph} % optional

% XeTeX
\usepackage[cm-default]{fontspec}
\usepackage{xltxtra,xunicode}
\defaultfontfeatures{Scale=MatchLowercase,Mapping=tex-text}
\setmainfont{Hoefler Text}
\newfontfamily{\sbl}{SBL BibLit}

% Bibliography
\usepackage[round]{natbib}

% Title Information
\title{\mytitle}
\author{\myauthor} 
\date{} % Leave blank for no date, comment out for most recent date

% PDF Stuff
\usepackage[plainpages=false, pdfpagelabels, bookmarksnumbered, backref, pdftitle={\mytitle}, pdfauthor={\myauthor}, pdfkeywords={\mykeywords}, xetex, colorlinks=true, citecolor=gray, linkcolor=gray, urlcolor=gray, unicode=true]{hyperref} 

%%% BEGIN DOCUMENT
\begin{document}

% Title Page
\maketitle
% \begin{abstract} % optional
% \noindent
% \end{abstract}
% \vskip 2em \hrule height 0.4pt \vskip 2em
% \epigraph{} % optional; make sure to uncomment \usepackage{epigraph}

% Layout Settings
\setlength{\parindent}{1em}

% Main Content

\section{On the Significance of Priscian's account} % (fold)
\label{sec:on_the_significance_of_priscian_s_account}

Priscian of Lydia's \emph{Metaphrasis in Theophrastum} has been a rich doxographical source for scholars interested in Theophrastus and Iamblichus. The \emph{Metaphrasis} is billed as a paraphrase of the fifth book of the now lost work \emph{Physics} by Aristotle's student and successor Theophrastus. We know from Themistius that the fourth and fifth books of \emph{Physica} concerned the soul. And from the \emph{Metaphrasis}, we know that it consisted, at least in part, in Theophrastus raising some questions concerning Aristotle's \emph{De anima}. In the \emph{Metaphrasis}, Priscian endeavours to answer these questions relying upon the psychological doctrines of Iamblichus. Priscian answers Theophrastus' questions \emph{in propria voce}. Priscian is principally concerned to set down the truth of the matter, as he understands it, rather than using Theophrastus' questions as an opportunity to engage in a closer exegesis of \emph{De anima}. What Aristotle or Theophrastus might have meant matters little, especially if it is potentially at variance with the truth of the matter as revealed by Iamblichus. (On Priscian's method as compared to Pseudo-Simplicius, see \citealt[7--10]{Steel:1978th}.)

The \emph{Metaphrasis}, in the fragmentary state that it has come down to us, begins with a puzzle or \emph{aporia} raised by Theophrastus concerning the formal assimilation involved in perception. If perception involves somehow becoming like the perceived object, then what does this becoming like consist in? ``For with sense-organs, and even more with the soul, the capacity to become like <an object> in color and tastes and sound and shape seems absurd'' (Priscian, \emph{Metaphrasis} 1 5; Pamela Huby in \citealt{Sorabji:1997ly}). This puzzle occasions Priscian's account of the process involved in perception's formal assimilation to its object. In his discussion, Priscian is principally concerned to establish that the soul makes itself like the perceived object actually is in a way that contrasts with the passive reception of an impression (\emph{Metaphrasis} 1.13--16, 3.7--8).

Priscian's account thus belongs to a general class of such accounts where the soul is the efficient cause of its likeness to the perceived object. Such accounts can be found among late Platonists such as Priscian and Pseudo-Simiplicius (possibly one and the same, see \citealt{Bossier:1972rp}, \citealt{Steel:1978th}, and Steel's introduction to his translation of \emph{In de anima} in \citealt[103--140]{Sorabji:1997ly}; for discussion see \citealt[18-24]{Finamore:2002yf}), Christian Platonists such as Augustine and the medieval thinkers that took inspiration from him, and among the thinkers involved in the Renaissance revival of Simplician Averroism (for discussion see \citealt[chapter 8]{Spruit:1995fh}). All such accounts face a general problem given that the sensible form of the perceived object is excluded as explanatorily relevant to the soul's formal assimilation, being confined to at best occasioning the soul's activity. This explanatory exclusion is the basis of related problems raised by, \emph{inter alia}, Plotinus in the opening \emph{aporia} of \emph{Ennead} 3.6, in Aquinas' criticism of Augustine in \emph{Quaestiones disputatae de veritate} 10.6, and Duns Scotus' criticism of Olivi in his \emph{Ordinatio} 1.3.3.4. The principle interest of Priscian's account is that it, along with Pseudo-Simiplicius' account in \emph{In de anima}, represents a way out of this general difficulty, though one that Aquinas judged absurd. My aim in the present essay is to set out Priscian's account of perception in the \emph{Metaphrasis} as clearly and sympathetically as I can, with an eye to what light it may shed on the related problems facing the general class of accounts to which Priscian's belongs.

% Aquinas characterizes the general class as follows:
% \begin{quote}
% 	Other<s> \ldots\ said that the soul is the cause of its own knowledge. For it does not receive knowledge from sensible things as if likenesses of things somehow reached the soul because of the activity of sensible things, but the soul itself, in the presence of sensible things, constructs in itself the likenesses of sensible things. (Aquinas, \emph{Quaestiones disputatae de veritate} 10.6; \citealt[24]{James-V.-McGlynn:1953rz})
% \end{quote}

% ; Jacopo Zabarella offers an interesting argument from selective attention against the Simplicianism of his teacher Marcantonnio Genua in \emph{Liber de densu agente}

% section on_the_significance_of_priscian_s_account (end)

\section{Theophrastus' \emph{Aporia}} % (fold)
\label{sec:theophrastus_emph_aporia}

The \emph{Metaphrasis}, as it comes down to us, begins as follows:
\begin{quote}
	His <Theophrastus'> next target is concerned with sense-perception. Since Aristotle wants the sense-organs, when moved by the objects of sense to become like those objects by being affected passively, he asks what the becoming like <consists in>. For with sense-organs, and even more with the soul, the capacity to become like <an object> in color and tastes and sound and shape seems absurd. Indeed he himself also says that the becoming like occurs with the regard to the forms and the \emph{logoi} without matter. (Priscian, Metaphrasis 1.3--8; Huby in \citealt{Sorabji:1997ly})
\end{quote}

Theophrastus' \emph{aporia} concerns the formal assimilation involved in perception. It is initially introduced with respect to the way the sense-organs become like the objects of perception when these act upon them. Aristotle does sometimes speak of sense-organs receiving the forms of perceptible objects (for example, \emph{De anima} 425\( ^{b} \)23--4, 435\( ^{a} \)22--4). On the standard Peripatetic account, the alteration of natural bodies involves one natural body acting upon another where the patient is, at the beginning of this process, potentially like the agent and, at the end of the process, is actually like it. Moreover, an object of perception acting upon a sense-organ such as to become like that object, in whatever relevant sense, can seem, at first blush, to be the kind of formal assimilation characteristic of natural bodies acting upon one another more generally. This, however, is misleading in at least two respects. 

First, as we shall see, Priscian will insist that the sense-organ is not a mere natural body but that life inheres in it. It is animated by the sensitive soul. Being animated makes a difference to how exactly it may be affected. The life that inheres in the eye contributes to the way in which it may be affected from without.

Second, the formal assimilation involved in perception is not confined to the way in which the sense-organ assimilates to the form of the perceived object that acts upon it. Priscian makes this clear in the final line of the quoted passage. Here we have an allusion to Aristotle's definition of perception as the assimilation of form without matter (\emph{De anima} 2.5 418\( ^{a} \)3--6, 2.12 424\( ^{a} \)18--23), though as Huby observes, it is unclear whether the ``he himself'' is meant ot refer to Aristotle or to Theophrastus restating the Aristotelian doctrine \citep[49--50 n11]{Sorabji:1997ly}. Whereas Aristotle is willing to speak of the sense-organ (\emph{aisthêtêrion}) as assimilating to the form of the perceived object, at \emph{De anima} 2.5 418\( ^{a} \)3--6, 2.12 424\( ^{a} \)18--23 Aristotle is characterizing perception (\emph{aisthêsis}) as a kind of formal assimilation. While Priscian will discuss the formal assimilation involved in the object of perception acting upon the perceiver's sense-organ, his main focus will be on the formal assimilation involved in perception.

The objects of perception have ``forms and \emph{logoi}''. As these forms are assimilated in perception, we may confidently assume that they are understood to be sensible forms. This assumption is confirmed by Priscian using whiteness as an example of such a form (\emph{Metaphrasis} 3.3). Priscian's talk of the \emph{logoi} of perceptible objects is arguably an interpretation of an occurence of \emph{logos} in \emph{De anima} 424\( ^{a} \)21--24: 
\begin{quote}
	{\sbl ὁμοίως δὲ καὶ ἡ αἴσθησις ἑκάστου ὑπὸ τοῦ ἒχοντος χρῶμα ἢ χυμὸν ἢ ψόφον πάσχει, ἀλλ᾽ οὐχ ᾗ ἓκαστον ἐκείνων λέγεται, ἀλλ᾽ ᾗ τοιονδί, καὶ κατὰ τὸν λόγον.}
\end{quote}
Aristotle's general claim here is relatively clear. The senses are affected by what has color, taste, or sound. The senses are affected by what has color, taste, or sound, not insofar as they are the kinds of things that they are said to be, whether essentially or accidentally, but only insofar as they possess the relevant sensible forms. It is this latter positive claim, {\sbl ἀλλ᾽ ᾗ τοιονδί, καὶ κατὰ τὸν λόγον}, that stands in need of interpretation. {\sbl τοιονδί} is a general term meant to cover colors, tastes, and sounds and is commonly used by Aristotle to denote the category of quality \citep[416]{Hicks:1907uq}. \citet[417]{Hicks:1907uq} reads the present occurence {\sbl λὸγος} as equivalent to {\sbl εἶδος} and so as adding nothing to the formula of receiving form without matter that precedes it. Priscian, in effect, denies the equivalence. The \emph{logos} of the perceived object is something distinct from its sensible form and explanatory of it.

Subsequent occurrences of \emph{logos} in the \emph{De anima} passage refer not to the objects of perception but to the perceptual capacity, such as at \emph{De anima} 424\( ^{a} \)28 where Aristotle discusses the \emph{logos} of the sense---here standardly translated as ``ratio''---being destroyed by overly strong stimulation. And Priscian himself will speak of \emph{logos} in connection with the substance of the sensitive soul, thus following, in his own manner, this second class of occurrences of \emph{logos} in the \emph{De anima} discussion. Although, the \emph{logoi} in the substance or essence of the sensitive soul are not ratios, as Priscian conceives of them, but are more like concepts (see Lautner's note on the corresponding usage in Pseudo-Simplicius, \citealt[214 n.14]{Sorabji:1997ly}).

So perception is meant to assimilate to the form (and \emph{logos}?) of its object. In what does this assimilation consist in, asks Theophrastus. For it is absurd to suppose that the sense-organ becomes white when viewing a white thing (though, notoriously, some commentators attribute such a view to Aristotle \citealt{Slakey:1961ss, Sorabji:1974fk,Everson:1997ep}, implausibly to my mind \citealt{Kalderon:2015fr}). And it is even more absurd to suppose that the soul becomes white when seeing a white thing (though this conclusion was embraced by William Crathorn in his commentary on Lombard's \emph{Sentences}: ``A soul seeing and intellectively cognizing color is truly colored,'' \emph{Quaestiones super librum sententiarum} q. 1 concl. 7 \citealt[288]{Pasnau:2002pb}, prompting Robert Holcot to compare the soul, as Crathorn conceived of it, to a chameleon; see \citealt[chapter 1.1]{Pasnau:1997aa} for discussion). But if neither the sense-organ nor the sensitive soul become like, in the most straightforward sense, the perceived object actually is, then in what does perception's formal assimilation consist in? That is the \emph{aporia} posed by Theophratus that occasions Prician's account of the process of formal assimilation in perception.

Perception, as Priscian conceives of it, is a mode of recognition. It is a distinctively perceptual mode of recognition, albeit as conceived by a Platonist. Specifically, perception involves the judgment (\emph{krisis}) and understanding (\emph{sunesis}) of the sensitive soul (\emph{Metaphrasis} 7.15--16). Of course, \emph{krisis} can mean discrimination in a sense that need not imply judgment, but, among the late Platonists, in discussions of perception, it typically has the more cognitively loaded sense of judgment. Though Steel understands \emph{krisis} as it occurs in Pseudo-Simiplicius as designating a non-rational mode of discrimination \citep[see Lautner's note in][222 n.131]{Sorabji:1997ly}. And since Steel thinks that Pseudo-Simplicius is Priscian, presumably he would take \emph{krisis} in the \emph{Metaphrasis} to mean discrimination as opposed to judgment as well. I agree with Steel that \emph{krisis} is the activity of the sensitive as opposed to the rational soul. So it is not the rational soul's judgment about the deliveries of the senses. Nevertheless, I am inclined to understand the activity of the sensitive soul in the more cognitively loaded sense of judgment for two reasons. First, it is natural to suppose that Priscian held to the Iamblichian doctrine that reason suffuses all things. If so, the activity of the sensitive soul would reflect, insofar as it can, the activity of the rational soul. Second, this activity involves the sensitive soul's projection of a \emph{logos} native to its substance or essence and fitting it to the appearance of the form of the perceived object. Insofar as recognition involves the application of concepts to the objects of awareness, then, whether or not \emph{logos} in this context is best understood as a concept, the activity of the sensitive soul as described by Priscian is a reasonable approximation, thus making the more cognitively loaded translation apt.

Recognition involves awareness. The awareness of the sensible and corporeal afforded by perception is conceived to be a mode of knowledge. All modes of knowledge involve gathering together the object of knowledge into an indivisible encompassment (\emph{Metaphrasis} 1.11--13, 2.12). What is the plurality that is gathered together into an indivisible encompassment? Where the mode of knowledge is sensory awareness, the plurality that is gathered together into an indivisible encompassment consists in the spatial and temporal parts of the perceived object:
\begin{quote}
	perception encompasses without division the beginning and the middle parts and the end of the sensed object, and is actuality and complete awareness and altogether as a whole in the present, and exists directly by way of the form of the sensed object. (Priscian, \emph{Metaphrasis} 2.12--14; Huby in \citealt{Sorabji:1997ly})
\end{quote}
Sensory awareness thus involves the encompassment of its object as whole and at the same time. If sensory awareness only encompassed its object part by part and over time, it would be incomplete and never fully actual. But the activity of the sensitive soul manifest in sensory awareness is complete at every moment and fully actual.

From this conception of perception as a mode of knowledge and understanding, Priscian draws a conclusion about the activity of the knower and the role it plays in the knower's assimilation to the object known. Specifically, Priscian maintains that the knower must be active in a manner corresponding to the form of the object known. The activity of the knower consists in the projection before itself of a \emph{logos} native to its substance or essence that is fitted to the form of the object known. In the case of perception, the activity of the sensitive soul corresponds to the form of the object of perception. It does so by projecting a \emph{logos} native to its substance or essence before itself that is fitted to the appearance of the form of the perceived object. However, in contrast with rational modes of knowledge, this activity is only ever occasioned by the affection (\emph{pathêma}) of the relevant sense-organ.

So the sensitive soul becomes like the perceived object actually is, in the sense that it does, in a manner that contrasts with the passive reception of an impression. The process by which perception is perfected and so becomes like the perceived object actually is is complex. At every stage of this complex process, Priscian is keen to emphasize the activity of the sensitive soul to minimize the role of passive affection from without. His principle aim, in his discussion of Theophrastus' \emph{aporia}, is to demonstrate that while perception may be occasioned by the presence of the form and \emph{logos} of an external body, nevertheless it is the sensitive soul that makes itself like the external form through an activity, aroused from within itself, that corresponds to it.



% section theophrastus_emph_aporia (end)

\section{Perception's Formal Assimilation} % (fold)
\label{sec:perception_s_formal_assimilation}

On Priscian's account, the sensitive soul is active in three moments in the process of perception's formal assimilation to its object. Only in the third moment is perception perfected and the soul is made like that object. (In conceiving of the process as a progressive perfection, Priscian's approximates a pattern in late Platonist accounts of perception, see \citealt[142]{Lloyd:1990dp}.) Unlike an impression on wax, the external form is never directly received. In the first moment, what is received in the sense-organ is, not the external form, but its effect, a motion in the sense-organ. Though the sense-organ is only affected in the manner in which it is thanks to its own vital activity being animated by the sensitive soul. In the second moment, this effect is perfected into a form in the life that inheres in the sense-organ. The perfection of the effect into a form in the life of the sense-organ is itself the activity of the sensitive soul. In the third moment, a \emph{logos} native to the substance or essence of the sensitive soul is fitted to the vital form. The sensitive soul becomes like the perceived object actually is in the sense that its activity in this way corresponds to the form of the perceived object. As Priscian emphasizes, it is external bodies and their sensible forms that are the objects of sensory awareness and not their effects on our sense-organs:
\begin{quote}
	the objects of sense are outside: for sense-perception is of these and not of the effects in the sense-organs, but together with these it grasps the forms in the <external> bodies. (Priscian, \emph{Metaphrasis} 1.24--2.1; Huby in \citealt[9]{Sorabji:1997ly})
\end{quote}

1. \emph{First moment}. The first moment, involving the reception of the effect of the external form, is the most passive in the process of formal assimilation. Importantly, however, it is not altogether passive. ``It is not like the soulless things that the sense-organs are affected by sense-objects, but as a living body is affected'' (Priscian, \emph{Metaphrasis} 2.1--2; Huby in \citealt[9--10]{Sorabji:1997ly}). Being animated makes a difference to how exactly the sense-organ may be affected. The life that inheres in the eye contributes to the way in which it may be affected from without. A dead eye may be affected from without just as much as a living eye of a whole and healthy human perceiver. The interior of the dead eye may be illuminated, say, but it won't be affected in the same way as the living eye. It won't be subject, for example, to a pattern of retinal stimulation. 

According to Priscian, the perceived object acts directly on the sense-organ. While a medium must be postulated to explain the operation of sight, Priscian departs from the standard Peripatetic account, where the distal object acts directly upon the medium, and the medium, in turn, acts directly upon the sense-organ. Priscian (\emph{Metaphrasis}) evidently shared Plotinus' (\emph{Ennead}) concern that the medium, so conceived, would screen off the distal object in sense experience. Rather, the medium carries the activity of the distal object unmixed thus affording the object direct causal access to the sense-organ.

2. \emph{Second moment}.

3. \emph{Third moment}.

% section perception_s_formal_assimilation (end)

\section{Some Related Problems \ldots} % (fold)
\label{sec:some_related_problems_ldots}

% section some_related_problems_ldots (end)

\section{\ldots and Priscian's Solution} % (fold)
\label{sec:_ldots_and_priscian_s_solution}

% section _ldots_and_priscian_s_solution (end)

% No cite

\bibliographystyle{plainnat}
\bibliography{Philosophy}

\end{document}
