%!TEX TS-program = xelatex 
%!TEX TS-options = -synctex=1 -output-driver="xdvipdfmx -q -E"
%!TEX encoding = UTF-8 Unicode
%
%  priscian
%
%  Created by Mark Eli Kalderon on 2016-07-26.
%  Copyright (c) 2016. All rights reserved.
%

\documentclass[12pt]{article} 

% Definitions
\newcommand\mykeywords{Priscian, perception}
\newcommand\myauthor{Mark Eli Kalderon}
\newcommand\mytitle{Priscian on Perception}

% Packages
\usepackage{geometry} \geometry{a4paper} 
\usepackage{url}
% \usepackage{txfonts}
\usepackage{color}
\usepackage{enumerate}
\definecolor{gray}{rgb}{0.459,0.438,0.471}
% \usepackage{setspace}
% \doublespace % Uncomment for doublespacing if necessary
% \usepackage{epigraph} % optional

% XeTeX
\usepackage[cm-default]{fontspec}
\usepackage{xltxtra,xunicode}
\defaultfontfeatures{Scale=MatchLowercase,Mapping=tex-text}
\setmainfont{Hoefler Text}
\newfontfamily{\sbl}{SBL BibLit}

% Bibliography
\usepackage[round]{natbib}

% Title Information
\title{\mytitle}
\author{\myauthor}
\date{} % Leave blank for no date, comment out for most recent date

% PDF Stuff
\usepackage[plainpages=false, pdfpagelabels, bookmarksnumbered, backref, pdftitle={\mytitle}, pdfauthor={\myauthor}, pdfkeywords={\mykeywords}, xetex, colorlinks=true, citecolor=gray, linkcolor=gray, urlcolor=gray, unicode=true]{hyperref} 

%%% BEGIN DOCUMENT
\begin{document}

% Title Page
\maketitle
\begin{abstract} % optional
\noindent An \emph{aporia} posed by Theophrastus prompts Priscian to describe the process by which perception formally assimilates to its object as a progressive perfection. I present an interpretation of Priscian's account of perception's progressive perfection. And I consider a dilemma for the general class of accounts to which Priscian's belongs based on related problems raised by Plotinus and Aquinas. Doing so reveals the explanatory structure of Priscian's account and the limitations of the general class of accounts to which Priscian's belongs.
\end{abstract}
% \vskip 2em \hrule height 0.4pt \vskip 2em
% \epigraph{} % optional; make sure to uncomment \usepackage{epigraph}

% Layout Settings
\setlength{\parindent}{1em}

% Main Content

\section{On the Significance of Priscian's account} % (fold)
\label{sec:on_the_significance_of_priscian_s_account}

Priscian of Lydia's \emph{Metaphrasis in Theophrastum} has been a rich doxographical source for scholars interested in Theophrastus and Iamblichus. The \emph{Metaphrasis} is billed as a paraphrase of the fifth book of the now lost work \emph{Physics} by Aristotle's student and successor at the Lyceum Theophrastus. We know from Themistius (\emph{In de anima} 3.5 108) that the fifth book of the \emph{Physics} concerned the soul. And from the \emph{Metaphrasis}, we know that it consisted, at least in part, in Theophrastus raising some questions concerning Aristotle's \emph{De anima}. In the \emph{Metaphrasis}, Priscian endeavours to answer these questions relying upon the psychological doctrines of Iamblichus. Priscian answers Theophrastus' questions \emph{in propria voce}. Priscian is principally concerned to set down the truth of the matter, as he understands it, rather than using Theophrastus' questions as an opportunity to engage in a closer exegesis of \emph{De anima}. What Aristotle or Theophrastus might have meant matters little, especially if it is potentially at variance with the truth of the matter as revealed by Iamblichus. (On Priscian's method as compared to Pseudo-Simplicius, see \citealt[7--10]{Steel:1978th}.)

The \emph{Metaphrasis}, in the fragmentary state that it has come down to us, begins with a puzzle or \emph{aporia} raised by Theophrastus concerning the formal assimilation involved in perception. If perception involves somehow becoming like the perceived object actually is, then in what does this becoming like consist? ``For with sense-organs, and even more with the soul, the capacity to become like <an object> in color and tastes and sound and shape seems absurd'' (Priscian, \emph{Metaphrasis} 1 5; Huby in \citealt{Sorabji:1997ly}). This puzzle occasions Priscian's account of the process of progressive perfection whereby perception formally assimilates to its object. In his discussion, Priscian is principally concerned to establish that the soul makes itself like the perceived object actually is in a way that contrasts with the passive reception of an impression (\emph{Metaphrasis} 1.13--16, 3.7--8).

Priscian's account thus belongs to a general class of such accounts where the soul is the efficient cause of its likeness to the perceived object. Such accounts can be found among late Platonists such as Priscian and Pseudo-Simiplicius (possibly one and the same, see \citealt{Bossier:1972rp}, \citealt{Steel:1978th}, and \citealt[103--140]{Sorabji:1997ly}; for discussion see \citealt[18-24]{Finamore:2002yf} and the references therein), Christian Platonists such as Augustine and the medieval thinkers that took inspiration from him \citep{Silva:2014bh,Toivanen:2013ul}, and among the thinkers involved in the Renaissance revival of Simplician Averroism (for discussion see \citealt[chapter 8]{Spruit:1995fh}). All such accounts face a dilemma given that the sensible form of the perceived object is excluded as explanatorily relevant to the soul's formal assimilation, being confined to at best occasioning the soul's activity. This explanatory exclusion is the basis of related problems raised by, \emph{inter alia}, Plotinus in the opening \emph{aporia} of \emph{Ennead} 3.6 and Aquinas in his criticism of Augustine in \emph{Quaestiones disputatae de veritate} 10.6. Together they constitute a dilemma facing the general class of accounts. The principle interest of Priscian's account is that it, along with Pseudo-Simiplicius' account in \emph{In de anima}, represents a way out of this general difficulty, though one that Aquinas judged absurd. My aim in the present essay is to set out Priscian's account of perception in the \emph{Metaphrasis} as clearly and sympathetically as I can, with an eye to what light it may shed on the dilemma facing the general class of accounts to which Priscian's belongs.

% Aquinas characterizes the general class as follows:
% \begin{quote}
% 	Other<s> \ldots\ said that the soul is the cause of its own knowledge. For it does not receive knowledge from sensible things as if likenesses of things somehow reached the soul because of the activity of sensible things, but the soul itself, in the presence of sensible things, constructs in itself the likenesses of sensible things. (Aquinas, \emph{Quaestiones disputatae de veritate} 10.6; \citealt[24]{James-V.-McGlynn:1953rz})
% \end{quote}

% ; Jacopo Zabarella offers an interesting argument from selective attention against the Simplicianism of his teacher Marcantonnio Genua in \emph{Liber de densu agente}

% section on_the_significance_of_priscian_s_account (end)

\section{Theophrastus' \emph{Aporia}} % (fold)
\label{sec:theophrastus_emph_aporia}

The \emph{Metaphrasis}, as it comes down to us, begins as follows:
\begin{quote}
	His <Theophrastus'> next target is concerned with sense-perception. Since Aristotle wants the sense-organs, when moved by the objects of sense to become like those objects by being affected passively, he asks what the becoming like <consists in>. For with sense-organs, and even more with the soul, the capacity to become like <an object> in color and tastes and sound and shape seems absurd. Indeed he himself also says that the becoming like occurs with the regard to the forms and the \emph{logoi} without matter. (Priscian, \emph{Metaphrasis} 1.3--8; Huby in \citealt{Sorabji:1997ly})
\end{quote}

Theophrastus' \emph{aporia} concerns the formal assimilation involved in perception. It is initially introduced with respect to the way the sense-organs become like the objects of perception when these act upon them. Aristotle does sometimes speak of sense-organs receiving the forms of perceptible objects (for example, \emph{De anima} 425\( ^{b} \)23--4, 435\( ^{a} \)22--4). On the standard Peripatetic account, the alteration of natural bodies involves one natural body acting upon another where the patient is, at the beginning of this process, potentially like the agent and, at the end of the process, is actually like it. Moreover, an object of perception acting upon a sense-organ such as to become like that object, in whatever relevant sense, can seem, at first blush, to be the kind of formal assimilation characteristic of natural bodies acting upon one another more generally. This, however, is misleading in at least two respects. 

First, as we shall see, Priscian will insist that the sense-organ is no soulless natural body but that life inheres in it. It is animated by the sensitive soul (the sensitive soul is not a numerically distinct soul, but a power or cluster of powers associated with a single entity). Being animated makes a difference to how exactly it may be affected. The life that inheres in the eye, or more specifically its vital activity, contributes to the way in which it may be affected from without.

Second, the formal assimilation involved in perception is not confined to the way in which the sense-organ assimilates to the form of the perceived object that acts upon it. Priscian makes this clear in the final line of the quoted passage. Here we have an allusion to Aristotle's definition of perception as the assimilation of form without matter (\emph{De anima} 2.5 418\( ^{a} \)3--6, 2.12 424\( ^{a} \)18--23), though as Huby observes, it is unclear whether the ``he himself'' is meant to refer to Aristotle or to Theophrastus restating Aristotelian doctrine \citep[49--50 n11]{Sorabji:1997ly}. Whereas Aristotle is willing to speak of the sense-organ (\emph{aisthêtêrion}) as assimilating to the form of the perceived object, at \emph{De anima} 2.5 418\( ^{a} \)3--6, 2.12 424\( ^{a} \)18--23 Aristotle is characterizing perception (\emph{aisthêsis}) as a kind of formal assimilation. While Priscian will discuss the formal assimilation involved in the object of perception acting upon the perceiver's sense-organ, his main focus will be on the formal assimilation involved in perception. The sense-organ's formal assimilation is merely an episode in a process of progressive perfection that is only complete when perception accords with perfect form.

The objects of perception have ``forms and \emph{logoi}''. As these forms are assimilated in perception, we may confidently assume that they are understood to be sensible forms. This assumption is confirmed by Priscian using whiteness as an example of such a form (\emph{Metaphrasis} 3.3). Priscian's talk of the \emph{logoi} of perceptible objects is arguably an interpretation of an occurence of \emph{logos} in \emph{De anima} 424\( ^{a} \)21--24. Aristotle's general claim, there, is relatively clear. The senses are affected by what has color, taste, or sound. The senses are affected by these, not insofar as they are the kinds of things that they are said to be, whether essentially or accidentally, but only insofar as they possess the relevant sensible forms. ``For example, when we see a man, the sense of sight is affected by him in so far as he is, say, white, and not because he is a rational, non-feathered biped'', \citet[113]{Hamlyn:2002ys} explains. It is this latter positive claim, {\sbl ἀλλ᾽ ᾗ τοιονδί, καὶ κατὰ τὸν λόγον}, that stands in need of interpretation. {\sbl τοιονδί} is a general term meant to cover colors, tastes, and sounds and is commonly used by Aristotle to denote the category of quality \citep[416]{Hicks:1907uq}. \citet[417]{Hicks:1907uq} understands {\sbl καὶ} as designating an explanatory relationship and reads the present occurence of {\sbl λὸγος} as equivalent to {\sbl εἶδος} and so as adding nothing further to the formula of receiving form without matter that immediately preceded it. Thus \citet[105]{Hicks:1907uq} translates the phrase as ``in so far as it possess a particular quality and in respect of its character or form''. Priscian denies the equivalence. Priscian, in effect, identifies {\sbl τοιονδί} as {\sbl εἶδος} and treats {\sbl λὸγος} as a distinct explanatory principle. The \emph{logos} of the perceived object is something distinct from its sensible form and explanatory of it. So understood, the \emph{logos} of the perceived object would be the intelligible principle underlying the occurence of its sensible form. Priscian's reading of this passage contrasts, in this way, not only with Hicks' reading but Ross'. \citet[264]{Ross:1961uq} understands {\sbl καὶ κατὰ τὸν λόγον} as ``in virtue of the relation to the sense in question.''

% :
% \begin{quote}
% 	{\sbl ὁμοίως δὲ καὶ ἡ αἴσθησις ἑκάστου ὑπὸ τοῦ ἒχοντος χρῶμα ἢ χυμὸν ἢ ψόφον πάσχει, ἀλλ᾽ οὐχ ᾗ ἓκαστον ἐκείνων λέγεται, ἀλλ᾽ ᾗ τοιονδί, καὶ κατὰ τὸν λόγον.}
% \end{quote}

% Subsequent occurrences of \emph{logos} in the \emph{De anima} passage refer not to the objects of perception but to the perceptual capacity, such as at \emph{De anima} 424\( ^{a} \)28 where Aristotle discusses the \emph{logos} of the sense---here standardly translated as ``ratio''---being destroyed by overly strong stimulation. According to Priscian, Theophrastus understands this ratio as characterizing, not a relationship among the elements that constitute the sense-organ, but a relationship between the sense-organ and the perceived object, such as the appropriate distance the eye must be from a particular body in order to clearly view it (\emph{Metaphrasis} 21.7--9). Priscian himself will speak of \emph{logos} in connection with the substance of the sensitive soul (\emph{Metaphrasis} 21.9--12), thus following, in his own manner, this second class of occurrences of \emph{logos} in the \emph{De anima} discussion. However, the \emph{logoi} in the substance or essence of the sensitive soul are not ratios, as Priscian conceives of them, but are more like concepts (see Lautner's note on the corresponding usage in Pseudo-Simplicius, \citealt[214 n.14]{Sorabji:1997ly}).

Perception assimilates to the form (and \emph{logos}?) of its object. How are we to understand this? For it is absurd to suppose that the sense-organ becomes white when viewing a white thing (though, notoriously, some commentators attribute such a view to Aristotle, \citealt{Slakey:1961ss, Sorabji:1974fk,Everson:1997ep}). And it is even more absurd to suppose that the soul becomes white when seeing a white thing (though this conclusion was embraced by William Crathorn in his commentary on Lombard's \emph{Sentences}: ``A soul seeing and intellectively cognizing color is truly colored,'' \emph{Quaestiones super librum sententiarum} q. 1 concl. 7 \citealt[288]{Pasnau:2002pb}). But if neither the sense-organ nor the sensitive soul become like, in the most straightforward sense, the perceived object actually is, then in what does perception's formal assimilation consist? How can perception's formal assimilation to its object be understood so as to avoid these two absurd alternatives? That is the \emph{aporia} posed by Theophratus that occasions Prician's account of the process of perception's formal assimilation to its object.

Perception, as Priscian conceives of it, is a mode of recognition. It is a distinctively perceptual mode of recognition, albeit as conceived by a Platonist. Specifically, perception involves the judgment (\emph{krisis}) and understanding (\emph{sunesis}) of the sensitive soul (\emph{Metaphrasis} 7.15--16). Note well that it is the judgment and understanding of the sensitive, as opposed to the rational, soul. Of course, \emph{krisis} can mean discrimination in a sense that need not imply judgment, but, among the late Platonists, in discussions of perception, it typically has the more cognitively loaded sense of judgment. Steel, however, understands \emph{krisis} as it occurs in Pseudo-Simiplicius as designating a non-rational mode of discrimination \citep[see Lautner's note in][222 n.131]{Sorabji:1997ly}. And since Steel thinks that Pseudo-Simplicius is Priscian, presumably he would take \emph{krisis} in the \emph{Metaphrasis} to mean discrimination as opposed to judgment as well. I agree with Steel that \emph{krisis} is the activity of the sensitive as opposed to the rational soul. So it is not the rational soul's judgment about the deliveries of the senses. Nevertheless, I am inclined to understand the activity of the sensitive soul in the more cognitively loaded sense of judgment for two reasons. First, it is natural to suppose that Priscian held the Iamblichean doctrine that reason suffuses all things. If so, the activity of the sensitive soul would reflect, insofar as it can, the activity of the rational soul. Second, this activity involves the sensitive soul's projection of a \emph{logos} native to its substance or essence and fitting it to the appearance of the form of the perceived object. (As we shall see, Prisican's talk of projection, here, derives from Proclus.) Insofar as recognition involves the application of concepts to the objects of awareness, then, whether or not \emph{logos} in this context is best understood as a concept, the activity of the sensitive soul as described by Priscian is a reasonable approximation, thus making the more cognitively loaded translation apt.

Recognition involves awareness. Perception affords awareness of the sensible forms of external bodies. The awareness of the sensible forms of bodies afforded by perception is conceived to be a mode of knowledge. All modes of knowledge involve gathering together the object of knowledge into an indivisible encompassment (\emph{Metaphrasis} 1.11--13, 2.12). This indivisible encompassment is incorporeal, as corporeal encompassments, such as grasping a stone in one's fist, are divisible (\emph{Metaphrasis} 22.8--9). What is the plurality that is gathered together into an indivisible encompassment? Is it the knower and the object known? Whatever may be the case, where the mode of knowledge is sensory awareness, the plurality that is gathered together into an indivisible encompassment consists at least in the spatial and temporal parts of the perceived object:
\begin{quote}
	perception encompasses without division the beginning and the middle parts and the end of the sensed object, and is actuality and complete awareness and altogether as a whole in the present, and exists directly by way of the form of the sensed object. (Priscian, \emph{Metaphrasis} 2.12--14; Huby in \citealt{Sorabji:1997ly})
\end{quote}
Sensory awareness thus involves the encompassment of its object as a whole and all at once (compare Plotinus \emph{Ennead} 3.6.18 24--29). If sensory awareness only encompassed its object part by part and over time, it would be incomplete and never fully actual. But the activity of the sensitive soul manifest in sensory awareness is complete at every moment, fully actual, and accords with perfect form.

% Prisician, here, is heir to a tradition, tracing back to Alcinous, of treating forms as the objects of knowledge, \emph{Didaskalikos} 163.14ff. 

% The whiteness immanent in a body may extend throughout its surface, for a certain passage of time, but that whiteness is presented as a whole and all at once in the awareness afforded by sight. 

From this conception of perception as a mode of knowledge and understanding, Priscian draws a conclusion about the activity of the knower and the role it plays in the knower's assimilation to the object known: 
\begin{quote}
	It is necessary, then, for that which knows to be in an active state corresponding to the form of the object known, and to have projected before itself the distinguishing mark <\emph{kharaktêr}> of the thing, and this is the becoming like. (Priscian, \emph{Metaphrasis} 1.12--14; Huby in \citealt[9]{Sorabji:1997ly})
\end{quote}
Specifically, Priscian maintains that the knower must be active in a manner corresponding to the form of the object known. The activity of the knower consists in the projection before itself of a \emph{logos} native to its substance or essence that is fitted to the form of the object known. In the case of perception, the activity of the sensitive soul corresponds to the form of the object of perception. It does so by projecting before itself a \emph{logos} native to its substance or essence that is fitted to the appearance of the form of the perceived object. However, in contrast with rational modes of knowledge, this activity is only ever occasioned by the affection (\emph{pathêma}) from without of the relevant sense-organ (\emph{Metaphrasis} 1.14--16). The external sensible body is in this way the \emph{causa occasionalis} of its perception \citep[112]{Lautner:1994cs}.

So the sensitive soul becomes like the perceived object actually is, in the sense that it does, in a manner that contrasts with the passive reception of an impression. The process by which perception is perfected and so becomes like the perceived object actually is is complex. At every stage of this complex process, Priscian is keen to emphasize the activity of the sensitive soul to minimize the role of passive affection from without. His principle aim, in his discussion of Theophrastus' \emph{aporia}, is to demonstrate that while perception may be occasioned by the presence of the form and \emph{logos} of an external body, nevertheless, it is the sensitive soul that makes itself like the external form through an activity, aroused from within itself, that corresponds to it. Moreover, in so doing, Priscian takes himself to have resolved Theophrastus' \emph{aporia}.

% section theophrastus_emph_aporia (end)

\section{Perception's Progressive Perfection} % (fold)
\label{sec:perception_s_formal_assimilation}

On Priscian's account, the sensitive soul is active in three moments in the process of progressive perfection whereby perception formally assimilates to its object. In conceiving of the process as a progressive perfection, Priscian's approximates a pattern in late Platonist accounts of perception (see \citealt[142]{Lloyd:1990dp}). The first two moments concern the formal assimilation of the sense-organ to the object of perception. Only in the third moment is perception perfected and the soul is made like that object. There is a further contrast. Whereas the first two moments involve the activity of the compound or living being, the third moment involves only the activity of the sensitive soul apart from the body. Finally, unlike a corporeal impression, the external form is never directly received. 

% (\emph{aisthêtêrion}) (\emph{aisthêsis})

The first moment is a moment of reception. In the first moment, what is received in the sense-organ is, not the external form, but its effect, a motion in the sense-organ, though the sense-organ is only affected in the manner in which it is thanks to the activity of the sensitive soul that animates it. The second moment is a moment of refinement. In the second moment, this effect is perfected into a form in the life that inheres in the sense-organ. The perfection of the effect into a vital form is itself the activity of the sensitive soul. The third moment is a moment of recognition. In the third moment, a \emph{logos} native to the substance or essence of the sensitive soul is fitted to the vital form aroused. The sensitive soul becomes like the perceived object actually is in the sense that its activity in this way corresponds to the form of the perceived object. 

Though the \emph{logoi} are fitted to forms perfected in the life of the sense-organ, as Priscian emphasizes, it is sensible forms of external bodies that are the objects of sensory awareness and not their effects on our sense-organs:
\begin{quote}
	the objects of sense are outside: for sense-perception is of these and not of the effects in the sense-organs, but together with these it grasps the forms in the <external> bodies. (Priscian, \emph{Metaphrasis} 1.24--2.1; Huby in \citealt[9]{Sorabji:1997ly})
\end{quote}
As we shall see, a dilemma facing the general class of accounts to which Priscian's belongs puts pressure on the purported objectivity of perception.

\subsection{The First Moment} % (fold)
\label{sub:the_first_moment}

The first moment is a moment of reception. The first moment, involving the reception of the effect of the external form, is the most passive in the process of progressive perfection. It involves the perceived object acting upon the sense-organ. 

According to Priscian, the perceived object acts directly upon the sense-organ. While a medium must be postulated to explain the operation of sight, Priscian departs from the standard Peripatetic account, where the distal object acts indirectly upon the sense-organ by acting directly upon the medium, and the medium, in turn, acting directly upon the sense-organ (on Priscian on the Peripatetic medium see \citealt{Ganson:2002aa}). Priscian (\emph{Metaphrasis} 12.10--14) evidently shared Plotinus' (\emph{Ennead} 4.5.2 50–55) concern that the medium, so conceived, would screen off the distal object in sense experience. Rather, the medium carries the activity of the distal object unmixed thus affording the object direct causal access to the sense-organ (\emph{Metaphrasis} 12.20--29). The way in which the activity of the perceived object is carried unmixed by the medium is plausibly an interpretation of the \emph{Timaeus} 45bff account of vision. Specifically, Priscian can be understood, here, as explaining the way in which, in the \emph{Timaeus}, the compound body, the continuous unity composed of fiery emanation and external light, passes on the motion of the perceived object to the soul through the eye.

The first moment, while the most passive moment in the process of progressive perfection is, importantly, however, not altogether passive. ``It is not like the soulless things that the sense-organs are affected by sense-objects, but as a living body is affected'' (Priscian, \emph{Metaphrasis} 2.1--2; Huby in \citealt[9--10]{Sorabji:1997ly}). Being animated makes a difference to how exactly the sense-organ may be affected. The life that inheres in the eye contributes to the way in which it may be affected from without. A dead eye may be affected from without just as much as a living eye of a whole and healthy human perceiver. The interior of the dead eye may be illuminated, say, but it would not be affected in the same way as the living eye. It would not be subject, for example, to a pattern of retinal stimulation. 

The vital activity of the sense-organ and the object of perception that acts upon it jointly cause an effect received in the sense-organ. Following Theophrastus, Priscian describes this effect as a motion. The motion, the joint effect of the external form and the activity that it aroused in the sense-organ, is a likeness (\emph{homoiôma}) of the sensible form of the external body. This likeness constitutes an \emph{emphasis} of the form of the perceived object. \emph{Emphasis} can mean appearance or reflection or even an appearance in a reflection. Huby in \citet[51 n.25]{Sorabji:1997ly} translates \emph{emphasis} as ``representative image''. It does assume a technical significance in Priscian's treatise. And nothing like what we would ordinarily describe as appearance has been achieved in the first moment (perhaps only in the second). Nor is Priscian explaining perception in terms of reflection. So perhaps Huby's non-committal rendering is apt.

Already in the first moment, the most passive moment in the process of progressive perfection, Priscian is keen to emphasize the activity of the sensitive soul and so minimize the role of passive affection from without: ``Hence neither is the whole thing <the motion> a passive effect nor is it entirely from outside, but is also by way of <the senses> own activity'' (Priscian, \emph{Metaphrasis} 2.4; Huby in \citealt[10]{Sorabji:1997ly}). Indeed, the joint determinants of the motion in the sense-organ are carefully counterposed: 
\begin{quote}
	and it is not the case that it <the sense> is moved first and is active later, but it is not moved at all without at the same time being active. And further it is not active without being moved. (Priscian, \emph{Metaphrais} 2.4--5; Huby in \citealt[10]{Sorabji:1997ly})
\end{quote}
Priscian evidently conceives of the first moment of the process of progressive perfection as a mode of arousal.

% Priscian may have been motivated to reduce the role of passive affection from without in the way that he does by reflection on the \emph{Timaeus} 45bff account of vision. Fire that gives light but does not burn emanates from the eye in a compressed stream. Since it is similar to it, the fiery emanation unites with the external light to constitute a continuous unity that connects the eye with the perceived object. Since it is a continuous unity, the compound of fiery emanation and external light passes on the motion of the object of perception through the body to the soul. The compound body is conceived as an instrument for the visual power of perception at a distance. This account was widely influential in the ancient world \citep[ch. 1]{Lindberg:1977aa}. It is arguably the source for Aristotle's conception of the medium as well as the stick analogy that Alexander of Aphrodisias attributes to the Stoics (\emph{De anima} 130.14). What may have impressed Prisician is the way in which the emanative activity of the living eye makes possible the reception of the motion of the perceived object.

Though not purely passive, the reception of the effect from the perceived object that constitutes its \emph{emphasis} is not yet perception (\emph{Metaphrasis} 2.9--10). Priscian gives three arguments. First, it remains passive rather than wholly active, and it is corporeal, divisible, and extended in time (\emph{Metaphrasis} 2.11). We may conclude that, for Priscian, perception, when perfected, is wholly active, incorporeal, an indivisible encompassment, and does not unfold through time. It encompasses its object as a whole and all at once. Second, Priscian thinks that the sense-organ may be affected in this way by the sense object and yet the perceiver not be conscious of it, as when they are asleep or awake and distracted (\emph{Metaphrasis} 2.16--17; compare Plotinus \emph{Ennead} 1.4.10). Finally, the object of perception is the sensible form of the external body, but Priscian denies that what has been received is a form (\emph{Metaphrasis} 2.19). This last argument is particularly significant. If what has so far been received falls short of being the external form, then since the external form is the object of perception, the object of perception is not passively received. As in modern scientific theories of perception, the proximal stimulant underdetermines the percept. That is why subsequent perfection is needed. These arguments are also relevant to Theophrastus' \emph{aporia} with which we began. If the motion in the sense-organ, which is a likeness of the external body, is not a form, then the sensible form of the external body is not in the likeness, which means that sense-organ, when affected from without, does not take on the form of the external body that acts upon it, thus so far avoiding the first of the two absurd alternatives of Theophrastus' \emph{aporia}. 

% subsection the_first_moment (end)

\subsection{The Second Moment} % (fold)
\label{sub:the_second_moment}

If the first moment was a moment of reception, albeit mediated by vital activity, the second moment is a moment of refinement. The object of perception is the sensible form of the external body. But what has been received is, at best, a corporeal likeness. So the presentation of this corporeal likeness, the \emph{emphasis} of the perceived object, is perfected into a form by the life and activity of the sense-organ as animated by the sensitive soul. Prisician emphasizes that the perfected form is in the life of the sense-organ and consists in its activity: ``But obviously the form by which sense-perception occurs is indeed in life, in that which consists in activity'' (\emph{Metaphrasis} 2.23--24; Huby in \citealt[10]{Sorabji:1997ly}). Nevertheless, this activity that constitutes the perfected form remains divided about the sense-organ (\emph{Metaphrasis} 2.27-28).

How does this second moment fare with Theophrastus' \emph{aporia}? The second moment, like the first, concerns the sense-organ's formal assimilation to the object of perception, as opposed to perception's formal assimilation. That occurs only in the third, terminal, moment of the process. Recall, the absurd alternative to be avoided is that the sense-organ takes on the perceived form such that the eye becomes white in seeing white things. If the likeness constituted by the motion in the sense-organ is perfected into a form in the life of that organ, then how is this absurd alternative avoided? The sensible form in the external body inheres in that body. It is a modification of that body and an affection. In contrast, the form perfected in the life of the sense-organ consists in its activity. So the perfected vital form is not a modification or affection the way the external form is. So the first of Theophrastus' absurd alternatives is avoided since in seeing a white thing, the eye does not, in this way, become white.

The arousal of vital form in the sense-organ might reasonably be described as a sensory appearance. At any rate, the sensitive soul's act of recognition that constitutes perception involves the fitting of a projected \emph{logos} to the vital form aroused. And that is a reasonable approximation of applying a concept to what appears in perceptual experience. (I do not say that \emph{logos} means concept, only that they are in some ways analogous. \emph{Logos} is, in this context, best understood as intermediary between the immanent sensible form and the the transcendental form of which it is an image.) Awareness of the object of appearance, however, only supervenes with the act of recognition involving the projection of \emph{logoi}. There is an interesting textual detail in Priscian's explanation of this. The perfected form, constituted by the vital activity of the sense-organ, is ``divided up around bodies and does not revert'' (see Huby in \citealt[51 n.31]{Sorabji:1997ly}). Perception has already been described as an indivisible encompassment. We now learn that this is accomplished through an act of reversion (\emph{epistrophê}), a kind of wholly folding within oneself (on reversion see \citealt[212--223]{Dodds:1963ul}, \citealt[126-30]{Lloyd:1990dp}; on reversion and sensory awareness see \citealt{Lautner:1994cs}). This provides an additional reason for the incorporeal character of the indivisible encompassment. Only that which is separate from bodies may revert (\emph{Metaphrasis} 22.5--6). Proclus provides the argument in his demonstration of proposition 15 of \emph{Elementatio Theologica}: ``All that is capable of reverting upon itself is incorporeal'':
\begin{quote}
	That which reverts upon anything is conjoined with that upon which it reverts: hence it is evident that every part of a body reverted upon itself must be conjoined with every other part---since self-reversion is precisely the case in which the reverted subject and that upon which is has reverted become identical. But this is impossible for a body, and universally for any divisible substance: for the whole of a divisible substance cannot be conjoined with the whole of itself, because of the separation of its parts, which occupy different positions in space. (Proclus, \emph{Elements of Theology} 15; \citealt[18--19]{Dodds:1963ul})
\end{quote}
So if the indivisible encompassment is accomplished through the sensitive soul's reversion, then this must be an incorporeal activity. 

% subsection the_second_moment (end)

\subsection{The Third Moment} % (fold)
\label{sub:the_third_moment}

If the first two moments concern the formal assimilation of the sense-organ to the object of perception, the third and final moment concerns the formal assimilation of perception to its object. If the first moment was a moment of reception, and the second a moment of refinement, the third moment is a moment of recognition. In it, the sensitive soul projects before itself a \emph{logos} native to its substance or essence and fits it to the vital form aroused. This is the act of recognition by which judgment (\emph{krisis}) and understanding (\emph{sunesis}) occur (\emph{Metaphrasis} 7.15--16).

The \emph{logos} fitted to the vital form is ``received beforehand by the soul'' (\emph{Metaphrasis} 2.29; Huby in \citealt[10]{Sorabji:1997ly}). It is part of the substance or essence (\emph{ousiôdês}) of the sensitive soul (\emph{Metaphrasis} 3.11). Part of the point of these claims is to contrast Priscian's rationalist epistemology with empiricist alternatives. In no sense is an idea of the form derived from its sensory presentation. The soul already contains within itself the \emph{logos} fitted to the vital form.

The \emph{logos} subsists in the soul and not the body (\emph{Metaphrasis} 2.34) and thus ``lives even of itself and is not only of the compound <of body and soul>'' (\emph{Metaphrasis} 2.29--30; Huby in \citealt[10]{Sorabji:1997ly}). This provides the basis of a contrast with the first two moments. The activity of the life of the sense-organ belongs to the compound as is evidenced by the fact that it is divided about the body. The activity of the \emph{logos}, however, pertains solely to soul apart from the body. This is why it is active undividedly (\emph{Metaphrasis} 2.31).

Priscian goes onto link the undivided activity of the \emph{logos} with its being ``cognitive of the objects of sense'' (\emph{Metaphrasis} 2.33; Huby in \citealt[10]{Sorabji:1997ly}). The \emph{logos}, though numerically one, is, by nature, a kind of generality. Though one it comprehends the many (\emph{Metaphrasis} 2.35--3.1). Thus the \emph{logos} of white fits each of the particular whites that we may perceive, and in perceiving each of them, the same \emph{logos} is fitted to the vital form aroused (\emph{Metaphrasis} 3.2--3). Insofar as the conceptual is a kind of generality, predicated of many things (\emph{De interpretatione} 7.17a 37--8), said of them but not in them (\emph{Categoriae} 2.1a 20--1b 9), Priscian's claim, here, befits the quasi-conceptual character of \emph{logoi} as he conceives of them. It is the undivided activity of the \emph{logos} that results in the incorporeal indivisible encompassment:
\begin{quote}
	for that which is aware is the \emph{logos}, and the synthesis connected with the sensitive soul, and the gathering together into the indivisible in the hypostasis separate from bodies. (\emph{Metaphrasis} 3. 6--8; ; Huby in \citealt[11]{Sorabji:1997ly})
\end{quote}
Again the plurality which is synthesized and gathered together into the indivisible are the spatial and temporal parts of the sensed object. Perception affords awareness of its object as a whole and all at once.

What is the connection, if any, between the \emph{logos} of the object of perception and the projected \emph{logos}? The matter is unclear. A tentative answer, however, may be found by beginning with another question. How does the \emph{logos} white, native to the substance or essence of the sensitive soul, unite all the particular white things such that this one \emph{logos}, given its nature as a generality, applies equally to all? Perhaps by picking out the \emph{logos} of white things, the intelligible principle that explains the occurence of their sensible form.

Perception is perfected by the projection of a \emph{logos} that is fitted to the vital form akin to it and that is itself a likeness of the external form (\emph{Metaphrasis} 3.3--6). This claim has several elements. These elements include (1) the projection of the \emph{logos}, (2) the \emph{logos} being akin to the vital form, (3) the \emph{logos} fitting the vital form, (4) and the fact that the vital form is a likeness of the external form. It will be useful to discuss these elements individually.

How are we to understand the projection of the \emph{logos} involved in sensory awareness consistent with it being an act of reversion? If projection (\emph{probolê}) is a kind of procession (\emph{proodos}), then it is a going out. But reversion (\emph{epistrophê}) is a turning in that contrasts with procession. On the face of it, then, the imagery suggests activities with conflicting directions. The difficulty is avoided if projection is not invariably understood to be a kind of procession that contrasts with reversion. Indeed, in the present instance, it is a moment in an act of reversion. (There is a similar usage in Pseudo-Simplicius, \emph{In de anima} 20.35--21.2, see Lautner's note in \citealt[164--165 n.94]{Urmson:2013vf}.) The \emph{logoi} subsist in the substance or essence of the sensitive soul. When they are projected, they are projected before the sensitive soul. Projection, here, is understood to be a kind of setting before the mind. (In this way, Priscian's use, in this context, of \emph{probolê} approximates a central aspect of Augustine's use of \emph{intentio}, \citealt[84--87]{ODaly:1987fq}.) An aspect of the sensitive soul's substance or essence thus becomes an object of its contemplation. And the sensitive's soul's contemplating an aspect of its own substance or essence might reasonably be described as an act of reversion.

Priscian's talk of projection derives from Proclus' refinement of a Meno-style epistemology in, \emph{inter alia}, his Euclid commentary (see \citealt[][for discussion]{Steel:1997ec}). The soul contains the \emph{logoi} of all things (proposition 194 of \emph{Elementatio Theologica}, though this doctrine is of an earlier provenance, see, for example, Porphyry \emph{Sententiae} 16). The soul possesses the logoi of all things because it is an image of the Intellect that contains the forms of all things (\emph{In primum Euclidis elementorum librum commnetarii} 16). Though to possess \emph{logoi} is to engage in cognitive activity, the \emph{logoi} internal to the soul are only articulated in an act where the \emph{logoi} are projected onto the imagination. (Think of a geometer working out a diagrammatic proof in imagination.) Priscian's innovation, in adapting this account to perception, is to reconceive the projection of the \emph{logoi} onto the screen of the imagination as a projection onto the sensorium, understood as the vital form arroused by the sensory object's effect on the sense organ.

The \emph{logos}, though distinct, is akin to the vital form aroused. Recall \emph{logoi} are intermediate between the immanent sensible form and the transcendental form. So the logos of white is an image of the form of whiteness and is distinguished from the whiteness immanent in a sensible body. Though distinct, the \emph{logos} may be said to be akin to the immanent sensible form that it applies to in that it has the distinguishing marks of that form (\emph{Metaphrasis} 1.14). These make the \emph{logos} applicable to this form, rather than another, with a different character, the sensible form of black, say. That the \emph{logos} is in this way akin to the vital form is a necessary precondition for its fitting.

The projected \emph{logos}, being akin to the vital form that satisfies its distinguishing marks, is fitted to it. And, at least as Priscian conceives of it, the satisfaction of the marks determined by the \emph{logos} by the vital form is a necessary precondition for that \emph{logos} to apply to that form.  

There is a small tension in Priscian's language here that is worth observing. Fitting is a corporeal image. That one thing fits another, a square peg fitting into a square hole, say, implies potential resistance. Such resistance is encountered when one attempts to turn the peg in its hole or vainly tries to fit the square peg into a circular hole. But talk of resistance is entirely out of place with respect to the intelligible (see, for example, Plotinus \emph{Ennead} 4.3.26 29–34 discussed below). As we shall see, this is linked with a tension in Priscian's account that arises at this point.

The vital form that the projected \emph{logos} is fitted to is a likeness of the external form. It is, after all, perfected in the life of the sense-organ from the corporeal likeness jointly determined by the vital activity of the sense-organ and the external form. This is relevant to Priscian's earlier insistence that the external form, and not the perceiver's body, is the object of sensory awareness (\emph{Metaphrasis} 1.24--2.1). If fitting the \emph{logos} to the vital form afforded awareness merely of that form, then what we would have would at best be an account of bodily sensation. For, recall, the perfected form consists in vital activity divided around the sense-organ. To be aware of such activity is to be aware of goings-on in the compound or living being. However, Priscian aims to account for perception, not bodily sensation. And the fact that the vital form perfected from the corporeal likeness retains the likeness of the external form is perhaps relevant here. The sensitive soul comes to be aware of the external form by fitting a \emph{logos} to the vital form which is a likeness and sign of that external form. The vital form thus plays a role akin to the role played by \emph{phantasma} in Aristotle's account of memory (\emph{De memoria et reminiscentia} 450\( ^{a} \)25–451\( ^{a} \)1).

So the perceiver is aware of the external form by their sensitive soul projecting before itself a \emph{logos} native to its substance or essence and fitting this \emph{logos} to a vital form that is a likeness and sign of the external form. In understanding sensory awareness as a mode of reversion, Priscian remarkably provides an account of perception on the model of self-knowledge (on reversion and self-knowledge see \citealt{Lautner:1994cs}). The soul is the cause of its knowledge since it constructs within itself a likeness of sensible things, occasioned by their presence, and it does so because it contains within itself the likeness of all things. Can the objectivity of perception be sustained on this basis?

Before turning to that question, consider first how the third and final moment fares with respect to Theophrastus' \emph{aporia}. As the third moment concerns the formal assimilation of perception to its object, the absurd alternative to be avoided is the one Crathorn embraced, that the soul takes on the sensible form of the external body. The sense in which the sensitive soul becomes like the perceived body is by engaging in activity that corresponds to its sensible form, namely, in the projection before itself of a \emph{logos} that fits that form. But the sensible form in the external body inheres in that body. It is a modification of that body and an affection. In contrast, the likeness in the sensitive soul consists in its activity. So perception's formal assimilation is not a modification or affection the way the external form is. And so the second of Theophrastus' absurd alternatives is avoided thus completing Prisican's resolution of Theophrastus' \emph{aporia}. At each of the moments in the process of progressive perfection, Theophrastus' absurd alternatives are avoided by an application of Aristotle’s distinction between \emph{kinêsis} and \emph{energeia} (\emph{De anima} 2.5), at least as Priscian understands that distinction.

% subsection the_third_moment (end)

% section perception_s_formal_assimilation (end)

\section{The Dilemma} % (fold)
\label{sec:some_related_problems_ldots}

That the soul contains within itself the \emph{logoi} of all things allows Priscian to understand sensory awareness as a mode of reversion, where the sensitive soul makes within itself a likeness of the external body. Can the objectivity of perception be sustained on this basis? We gain insight into the explanatory structure of Prisican's account, and the potential limits of the general class of accounts to which it belongs, by considering two related problems. The first problem is raised by Plotinus in \emph{Ennead} 3.6 and concerns whether perception so much as could have a content if it is not externally determined. That perception, as conceived by Prisician, lacks a determinate content would undermine its purported objectivity. The second problem, raised by Aquinas in \emph{Quaestiones disputatae de veritate} 10.6, grants that the content of perception could not be externally determined and concludes that if perception has a determinate content, it must be internally determined. But if the content of perception is internally determined, what guarantee is there that that it corresponds to anything external? Together, these problems constitute a dilemma facing the general class of accounts to which Priscian's belongs, namely, those where the soul makes within itself a likeness of an external body.

\subsection{Plotinus} % (fold)
\label{sub:plotinus}

Plotinus inaugurates \emph{Ennead} 3.6 with an \emph{aporia}:
\begin{quote}
	We stated that sense perceptions were not affections <\emph{pathê}>, but activities and judgments to do with impressions <\emph{pathêmata}>; affections are to do with something other than the soul---let us say body of such-and-such a kind---while the judgment is to do with the soul; it is not an affection, for if it were, we would need another judgment on it, and we would be involved in an infinite regress. Nevertheless we were faced with a problem here too---whether the judgment \emph{qua} judgment contained nothing of what was judged. True; if it were to take on some imprint <\emph{tupon}>, then it has been affected---although one could say even of the so-called imprints that they are made in a way quite different from what has been supposed, such as is found in thoughts, which are also activities able to discern without being affected in any way. (Plotinus, Ennead 3.6.1 1--14; \citealt[3]{Fleet:1995gf})
\end{quote}
The passage clearly refers to an earlier discussion for which, unfortunately, there is no surviving record. Since the matter has been previously discussed, Plotinus does not dwell on the details, and the inaugural \emph{aporia} serves merely as a device to introduce the principle theme of the treatise, the impassivity of the soul. (While the distinction between judgment and affection and the ``dematerialized'' notion of impression, as \citealt[292]{Dillon:2015yf} describe it, can be found elsewhere, they are linked with neither the regress argument nor the \emph{aporia}.) 

Whereas bodies are subject to affection, the soul is not, and its activities, such as judgment, must be understood as distinct from affections. Plotinus and late Platonists generally accept Aristotle's distinction between \emph{kinêsis} and \emph{energeia}, if not always as he understands it (though Plotinus expresses doubts about that distinction in \emph{Ennead} 6.1--3). And they accept the distinction even if they sometimes ungenerously disavow the attribution, as when Iamblichus complains that Aristotle fails to observe the distinction between motions in the category of change and motions in the category of life (\emph{De anima} 1; see \citealt[76--77]{Finamore:2002yf} for discussion). 

Sense-perceptions are not affections but activities and judgments having to do with impressions. Like Priscian after him, Plotinus maintains that it is external bodies and their sensible forms that are the objects of perception and not their effects on our sense-organs. Thus according to \citet[73]{Fleet:1995gf}, ``the judgment is not about what the impression \emph{is}, but what is \emph{of}'' (see also \citealt[75 n.28]{Emilsson:1988uq}; for a similar ambiguity in a parallel context in Augustine see \citealt{Brittain:2002hl}). If the judgment were merely about what the impression is, it could at best account for bodily sensation, not perception. 

The passage connects with the general class of accounts to which Priscian's belongs in the following way. Suppose that perception is, or at least involves, a mode of formal assimilation so that it becomes like, in some sense, the perceived object actually is. Since perceptions are not affections, the perceived object is not the efficient cause of the perception becoming like. So if perception involves formal assimilation, then the perception is made like the perceived object actually is by the activity of the sensitive soul. But that just is the distinctive claim of the general class of accounts.

The denial that perceptions are affections occasions the \emph{aporia}. Since that denial is common to the general class of accounts to which Priscian's belongs, the \emph{aporia} pertains to them generally. If perceptions are not affections, then they are not the effects of external corporeal form, and this raises a question about their very content that potentially undermines their purported objectivity. If perceptions are not affections but activities and judgments to do with impressions, then a question arises whether such judgments contained nothing of what was judged. For consider the content of a corporeal impression (here understood as \emph{tupos}), such as the impression made upon wax by a seal (\emph{Theaetetus} 194c--195a) or a signet ring (\emph{De anima} 2.12 424\( ^{a} \)18--23). The wax has in it the form impressed upon it by the seal. In contrast, a judgment, being an activity and not an affection, does not have in it a form impressed upon it by an external body. So how can it have a content or subject matter? The sensible form of the external body is meant to be the object of sensory awareness and so the subject matter of the judgment. But if judgment is not an affection and so has nothing in it of what is judged, then how is it a judgment at all?

Plotinus responds to the \emph{aporia} by pressing an analogy between perception and thought. In the cognitive domain, we may speak of impressions, if we like, but we must not understand them on the model of corporeal impressions. Specifically, corporeal impressions are affections, whereas cognitive impressions are activities. I take it that the reason it remains apt to speak of cognitive impressions is that, like their corporeal counterparts, cognitive impressions formally assimilate to what they are an impression of. The claim, then, is that the \emph{aporia} only arises on the assumption that cognitive impressions have their contents the way that corporeal impressions do, through affection. Plotinus elaborates on the ``dematerialized'' conception of cognitive impressions in \emph{Problems Concerning the Soul} (see also \emph{Ennead} 1.1.7 9ff):
\begin{quote}
	But first of all one would object that the impressions are not things with magnitude, nor are they like sealings, or resistances to pressure, or the making of impressions, because there is no pressing down, not even as in wax, but the way it happens is like intellection, even in the case of sense-objects; while in the case of intellections, on the other hand, what could on mean by resistance to pressure? And what need is there of a body or a bodily quality which goes along with it? (Plotinus, \emph{Ennead} 4.3.26 29--34; \citealt[101]{Dillon:2015yf})
\end{quote}
Again cognitive impressions differ from corporeal impressions in not being affec\-tions---``there is no pressing down''. And again we have the analogy between perception and thought. But not much by way of further elaboration.

The \emph{aporia} calls into question the very content of perception and so threatens its objectivity. It began with the claim that perception is not an affection caused by an external corporeal form. This claim may reasonably be generalized in the following manner---that the sensible form of the external body is explanatorily irrelevant to perception's formal assimilation to its object. So generalized, the problem becomes one of understanding how something explanatorily irrelevant to the character of perception could so much as be the object of its formal assimilation.

% subsection plotinus (end)

\subsection{Aquinas} % (fold)
\label{sub:aquinas}

Before considering how Priscian would address this problem, let us first consider a distinct, if related, problem raised by Aquinas. Aquinas' principle target is Augustine (compare, for example, the accounts of perception in \emph{De quantitate animae} and \emph{De musica}, see \citealt[208--210, n.73]{Colleran:1949ys}). Though Aquinas is clearly targeting Augustine, his discussion is couched in general terms, and he is explicitly criticizing a general class of views to which Augustine's (and Priscian's) belong:
\begin{quote}
	Other proponents \ldots\ said that the soul is the cause of its own knowledge. For it does not receive knowledge from sensible things as if likenesses of things somehow reached the soul because of the activity of sensible things, but the soul itself, in the presence of sensible things, constructs in itself the likenesses of sensible things. But this statement does not seem altogether reasonable. For no agent acts except in so far as it is in act. Thus, if the soul formed the likenesses of all things in itself, it would be necessary for the soul to have those likenesses of things actually within itself. (Aquinas, \emph{Quaestiones disputatae de veritate} 10.6; \citealt[24]{James-V.-McGlynn:1953rz})
\end{quote}
The first two lines of this passage clearly characterize the general class of accounts to which Priscian's belongs. And though they are not couched in Priscian's technical vocabulary, Priscian himself would undoubtedly assent to them. Priscian and Aquinas disagree, however, about the reasonableness of such an account. Aquinas begins by drawing out a consequence of any such account so characterized. If the external form is explanatorily irrelevant to the likeness of it in the soul, then it cannot be externally determined. But it is determined. So it must be internally determined. In order for the soul's likeness of the external corporeal form to be internally determined it must somehow already contain within itself that likeness. So the soul must contain beforehand the likeness of the external sensible form. But that, judges Aquinas, is absurd. (Aquinas may be echoing Aristotle's use of \emph{atopon} in \emph{Posterior Analytics} 2.19.) Again, while not couched in Priscian's technical vocabulary, the consequence that Aquinas draws at the very least approximates what Priscian explicitly endorses. And yet Priscian does not judge it to be absurd, at least in the form that he endorses.

We moderns should make an effort to determine the absurdity of this consequence, if it is absurd, and in what sense it is. After all, hasn't Chomsky made nativism scientifically respectable? I am reminded of the conception of perception that was something like the orthodoxy when I was a graduate student. According to it, perception was the tokening of a veridical mental representation that played the appropriate functional role. Notice that the mental representation type must in some sense subsist in the functional system, since that is defined in terms of its potential configurations, including the tokening of that mental representation. There is an important difference from Priscian's account, and one that Aquinas is sensitive to. On the functionalist conception, the mental representation exists beforehand merely in potentiality. Aquinas, however, is insisting that the internal likeness must be in actuality: ``For no agent acts except in so far as it is in act.'' And Priscian conceives of the \emph{logoi} as in actuality as well, subsisting in the substance or essence of the soul (compare Iamblichus' characterization of the soul as a \emph{plêrôma} \emph{logôn}, \emph{De anima} 7). In this way are the \emph{logoi} the source of their activity.

One worry, taking off from this, focuses on the distinctive nature of the sensitive soul. It is one thing to suppose that the intellect contains within itself everything intelligible, in the sense that intelligible objects subsist in actuality in the intellect (thus making Plotinus' comparison of the intellect with Kronos devouring his divine offspring apt, \emph{Ennead} 5.1.4 8--10, 5.1.7 33-34). It is another thing to suppose that the sensitive soul contains within itself, in the relevant sense, the \emph{logoi} of everything sensible. But that worry is only compelling once we have been offered sufficient grounds for distinguishing the intellect from the sensitive soul in this way. Moreover, there is reason to doubt whether such a distinction can be drawn in the required way in the context of an Iamblichean psychology that conceives of the soul as a mean between the intelligible and the sensible (Iamblichus \emph{De anima} 7, Pseudo-Simplicius \emph{In de anima} 5.39--6.18; for discussion see \citealt{Steel:1978th}, \citealt[91--93]{Finamore:2002yf}, and \citealt{Finamore:2014fc}).

The fundamental worry raised by the consequence that Aquinas judged absurd threatens the objectivity of perception, like Plotinus's \emph{aporia}, though in a different way. Bourke, in criticizing Augustine's account of perception, puts the worry vividly. Like Prisican, Augustine before him endeavoured to understand perception as the activity of the soul occasioned by an external body acting upon the perceiver's sense-organ. Concerning this \citet[112]{Bourke:1947jk} writes: ``One of its chief defects lies in its essential subjectivity. There is no natural guarantee that the representations which the soul makes within itself of the extra-mental world do truly correspond with physical events.'' Whereas Plotinus' \emph{aporia} threatened to undermine the objectivity of perception by calling into question its very content, Aquinas grants that perception has a determinate content but can be understood to call into question whether it corresponds to anything external. The explanatory exclusion of the external form can seem to rule out any such natural guarantee.

Consider further how Aquinas' problem is related to Plotinus'. They are, of course, distinct problems. Plotinus' problem makes no mention of internal likenesses, whereas Aquinas' problem turns on a commitment to internal likenesses. Nevertheless, they are importantly related. Aquinas, in focusing on what is, by his lights, the absurd consequence that the soul contains within itself the likeness of all things, highlights the way in which, as conceived by the general class of accounts, the external form is explanatorily irrelevant to the character of its perception. But that is what occasioned Plotinus' \emph{aporia}. Together, they constitute a dilemma facing the general class of accounts to which Priscian's belongs. If perception is not an affection, then it is not determined by the object of perception acting upon the sense-organ. But if perception is not determined by the object of perception, then either it lacks a determinate content, or if it has a determinate content, then it is internally determined. But if perception lacks a determinate content, then it involves the objective presentation of no object. And if it has an internally determined content, then there is no natural guarantee that the likeness that the soul makes within itself corresponds to anything external.

% subsection aquinas (end)


\subsection{Priscian's Solution?} % (fold)
\label{sec:_ldots_and_priscian_s_solution}

How does Prisican's account of perception fare with respect to the dilemma jointly posed by Plotinus and Aquinas?

Priscian and Pseudo-Simplicius clearly opt for the second horn of the dilemma and embrace the conclusion that Aquinas judged absurd. Priscian and Pseudo-Simplicius differ, in this way, from Augustine, who maintains that we cannot conceive of a sensible form without first perceiving it (\emph{De Trinitate} 13.8.14). Priscian accepts that perception has a determinate content, but not in the way that a corporeal impression has a content. To that extent, at least, he is in agreement with Plotinus. Since perception has a determinate content that is not externally determined the way that the content of a corporeal impression is, it must be internally determined. And so, in a sense, it is, on Priscian's account, since the soul contains within its substance or essence the \emph{logoi} projected in sensory awareness. If we bracket a blanket rejection of nativism, the potential problem facing Priscian's account is the problem that Bourke claims is facing Augustine's.

It would seem, however, that Priscian has provided an answer. Recall the \emph{logos} projected before the sensitive soul is fitted to the vital form aroused. While the vital form was perfected by the sensitive soul's activity, it is still to a degree passive, since it is the perfection of a corporeal likeness received. Priscian is thus in a position to respond as follows to Bourke's concern. The \emph{logoi} may be native to the substance or essence of the sensitive soul, but this does not preclude their objectivity. For in the activity involved in sensory knowledge, the projected \emph{logoi} are fitted to vital forms. It is this fit with that which is to a degree passive, being a perfection of the external form's effect, that provides a natural guarantee that the projected \emph{logos} corresponds with the sensible form of the external body.

Can Priscian, however, consistent with his own principles, endorse this reply, with its intended content? Insofar as the fitting of the \emph{logoi} to vital forms that are to a degree passive, being the perfection of the effects of external form, provides, in this way, a kind of external contraint, and so a natural guarantee that \emph{logoi} correspond with external form, fitting must be suitably understood. But can there be such a suitable understanding consistent with Priscian's principles? Perception, once perfected, is wholly active according to Priscian (\emph{Metaphrasis} 2.11). But how could this be if the activity that constitutes the indivisible encompassment consists in the fitting of the \emph{logoi} to that which is to a degree passive?

Recall the tension in Priscian's language, here. Fitting is a corporeal image. But corporeal fittings imply resistance in the way that incorporeal fittings could not. There are no ``resistances to pressure'' among the intelligible as Plotinus reminds us (\emph{Ennead} 4.3.26 29--34). Perhaps, just as Plotinus offered us a ``dematerialized'' conception of cognitive impressions, Priscian is offering us a ``dematerialized'' conception of fittings. Such a hypothesis is, I believe, plausible. Moreover, a ``dematerialized'' conception of a concept applying to an object is itself independently plausible. Frege subscribes to just such a conception. But what would provide external constraint, and so the wanted natural guarantee, is not just the \emph{logos} fitting to a form, but the \emph{logos} fitting to a presented form received, in part, from without. If the dilemma is to be fully answered, a ``dematerialized'' conception of fitting, not a form, but a received form needs to be explained. And the lingering Thomistic worry is that any external constraint provided by such a suitably ``dematerialized'' fitting would be too ethereal to secure the objectivity of perception. For if the activity of the \emph{logoi} pertain solely to the sensitive soul apart from bodies, what room is left for external constraint? 

If this difficulty could not be overcome, then even if perception involves an act of reversion, it could not wholly consist in such an act. If the application of internal \emph{logoi} are to be externally constrained so as to provide a natural guarantee that they correspond to external form, then it would seem that perception is better conceived as a procession, as the sensitive soul's departure from itself. Even if perception involves attending to a \emph{logos} in its own substance or essence, the sensitive soul must then fit that \emph{logos} to the vital form aroused, in part, from the external form. The sensitive soul, in thus subjecting itself to external constraint, would depart from itself. The sensitive soul, so conceived, may be active, but it would not be wholly active, at least if that involves the activity of the sensitive soul apart from bodies. Such an account, while not subject to the dilemma is not, however, consistent with the abstract principles that drive Priscian's.

This difficulty threatens, not only the objectivity of perception as Priscian conceives of it, but Priscian's resolution of Theophrastus' \emph{aporia} as well. Recall the second of Theophrastus' absurd alternatives---that the soul becomes white when seeing a white thing. Whiteness inheres in the external body as a modification or affection. But the whiteness does not inhere in the sensitive soul as a modification or affection. Rather, Prisician contends that perception's formal assimilation to its object is to be understood in terms of the sensitive soul's activity corresponding to the sensible form of the external body. But what it is for the sensitive soul's activity to correspond to the external form is for the projected \emph{logos} to be fitted to the vital form aroused, in part, from that form. But if a suitably ``dematerialized'' conception of fitting is problematic in the way suggested, then Priscian lacks a coherent account of how the soul's activity corresponds to external form, and hence his resolution of Theophrastus' \emph{aporia} is incomplete at best.

% section _ldots_and_priscian_s_solution (end)

% section some_related_problems_ldots (end)

% No cite

\bibliographystyle{plainnat}
\bibliography{Philosophy}

\end{document}
