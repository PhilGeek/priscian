%!TEX TS-program = xelatex 
%!TEX TS-options = -synctex=1 -output-driver="xdvipdfmx -q -E"
%!TEX encoding = UTF-8 Unicode
%
%  priscian
%
%  Created by Mark Eli Kalderon on 2016-07-26.
%  Copyright (c) 2016. All rights reserved.
%

\documentclass[12pt]{article} 

% Definitions
\newcommand\mykeywords{Priscian, perception}
\newcommand\myauthor{Mark Eli Kalderon}
\newcommand\mytitle{Priscian on Perception}

% Packages
\usepackage{geometry} \geometry{a4paper} 
\usepackage{url}
% \usepackage{txfonts}
\usepackage{color}
\usepackage{enumerate}
\definecolor{gray}{rgb}{0.459,0.438,0.471}
% \usepackage{setspace}
% \doublespace % Uncomment for doublespacing if necessary
% \usepackage{epigraph} % optional

% XeTeX
\usepackage[cm-default]{fontspec}
\usepackage{xltxtra,xunicode}
\defaultfontfeatures{Scale=MatchLowercase,Mapping=tex-text}
\setmainfont{Hoefler Text}

% Bibliography
\usepackage[round]{natbib}

% Title Information
\title{\mytitle}
\author{\myauthor} 
\date{} % Leave blank for no date, comment out for most recent date

% PDF Stuff
\usepackage[plainpages=false, pdfpagelabels, bookmarksnumbered, backref, pdftitle={\mytitle}, pdfauthor={\myauthor}, pdfkeywords={\mykeywords}, xetex, colorlinks=true, citecolor=gray, linkcolor=gray, urlcolor=gray, unicode=true]{hyperref} 

%%% BEGIN DOCUMENT
\begin{document}

% Title Page
\maketitle
% \begin{abstract} % optional
% \noindent
% \end{abstract}
% \vskip 2em \hrule height 0.4pt \vskip 2em
% \epigraph{} % optional; make sure to uncomment \usepackage{epigraph}

% Layout Settings
\setlength{\parindent}{1em}

% Main Content

\section{On the Significance of Priscian's account} % (fold)
\label{sec:on_the_significance_of_priscian_s_account}

Priscian of Lydia's \emph{Metaphrasis in Theophrastum} has been a rich doxographical source for scholars interested in Theophrastus and Iamblichus. The \emph{Metaphrasis} is billed as a paraphrase of the fifth book of the now lost work \emph{Physica} by Aristotle's student and successor Theophrastus. We know from Themistius that the fourth and fifth books of \emph{Physica} concerned the soul. And from the \emph{Metaphrasis}, we know that it consisted, at least in part, in Theophrastus raising some questions concerning Aristotle's \emph{De anima}. In the \emph{Metaphrasis}, Priscian endeavours to answer these questions relying upon the psychological doctrines of Iamblichus. Priscian answers Theophrastus' questions \emph{in propria voce}. Priscian is principally concerned to set down the truth of the matter, as he understands it, rather than using Theophrastus' questions as an opportunity to engage in a closer exegesis of \emph{De anima}. What Aristotle might have meant matters little, especially if it is potentially at variance with the truth of the matter as revealed by Iamblichus. (On Priscian's method as compared to Pseudo-Simplicius, see \citealt[7--10]{Steel:1978th}.)

The \emph{Metaphrasis}, in the fragmentary state that it has come down to us, begins with a puzzle or \emph{aporia} raised by Theophrastus concerning the formal assimilation involved in perception. If perception involves somehow becoming like the perceived object, then what does this becoming like consist in? ``For with sense-organs, and even more with the soul, the capacity to become like <an object> in color and tastes and sound and shape seems absurd'' (Priscian, \emph{Metaphrasis} 1 5; Pamela Huby in \citealt{Sorabji:1997ly}). This puzzle occasions Priscian's account of the process involved in perception's formal assimilation to its object. In his discussion, Priscian is principally concerned to establish that the soul is the efficient cause of its formal assimilation to the object of perception (\emph{Metaphrasis} 1.13--16, 3.7--8).

Priscian's account thus belongs to a general class of such accounts where the soul is the efficient cause of its likeness to the perceived object. Such accounts can be found among late Platonists such as Priscian and Pseudo-Simiplicius (possibly one and the same, see \citealt{Bossier:1972rp}, \citealt{Steel:1978th} Steel's introduction to his translation of \emph{In de anima} in \citealt[103--140]{Sorabji:1997ly}), Christian Platonists such as Augustine and the medieval thinkers that took inspiration from him, and among the thinkers involved in the Renaissance revival of Simplician Averroism (for discussion see \citealt[chapter 8]{Spruit:1995fh}). All such accounts face a general problem given that the sensible form of the perceived object is excluded as explanatorily relevant to the soul's formal assimilation, being confined to at best occasioning the soul's activity. This explanatory exclusion is the basis of related problems raised by, \emph{inter alia}, Plotinus in the opening \emph{aporia} of \emph{Ennead} 3.6, in Aquinas' criticism of Augustine in \emph{Quaestiones disputatae de veritate} 10.6, and Duns Scotus' criticism of Olivi in his \emph{Ordinatio} 1.3.3.4. The principle interest of Priscian's account is that it, along with Pseudo-Simiplicius' account in \emph{In de anima}, represents a way out of this general difficulty, though one that Aquinas judged absurd. My aim in the present essay is to set out Priscian's account of perception in the \emph{Metaphrasis} as clearly and sympathetically as I can, with an eye to what light it may shed on the related problems facing the general class of accounts to which Priscian's belongs.

% Aquinas characterizes the general class as follows:
% \begin{quote}
% 	Other<s> \ldots\ said that the soul is the cause of its own knowledge. For it does not receive knowledge from sensible things as if likenesses of things somehow reached the soul because of the activity of sensible things, but the soul itself, in the presence of sensible things, constructs in itself the likenesses of sensible things. (Aquinas, \emph{Quaestiones disputatae de veritate} 10.6; \citealt[24]{James-V.-McGlynn:1953rz})
% \end{quote}

% ; Jacopo Zabarella offers an interesting argument from selective attention against the Simplicianism of his teacher Marcantonnio Genua in \emph{Liber de densu agente}

% section on_the_significance_of_priscian_s_account (end)

\section{Theophrastus' \emph{Aporia}} % (fold)
\label{sec:theophrastus_emph_aporia}

The \emph{Metaphrasis}, as it comes down to us, begins as follows:
\begin{quote}
	His <Theophrastus'> next target is concerned with sense-perception. Since Aristotle wants the sense-organs, when moved by the objects of sense to become like those objects by being affected passively, he asks what the becoming like <consists in>. For with sense-organs, and even more with the soul, the capacity to become like <an object> in color and tastes and sound and shape seems absurd. Indeed he himself also says that the becoming like occurs with the regard to the forms and the \emph{logoi} without matter. (Priscian, Metaphrasis 1.3--8; Huby in \citealt{Sorabji:1997ly})
\end{quote}

Theophrastus' \emph{aporia} concerns the formal assimilation involved in perception. It is initially introduced with respect to the way the sense-organs become like the objects of perception when these act upon them. Aristotle does sometimes speak of the sense-organs as being receptive of the forms of the objects of perception (for example, \emph{De anima} 425\( ^{b} \)23--4, 435\( ^{a} \)22--4). On the standard Peripatetic account, the alteration of natural bodies involves one natural body acting upon another where the patient is, at the beginning of this process, potentially like the agent and, at the end of the process, is actually like it. Moreover, an object of perception acting upon a sense-organ such as to become like that object, in whatever relevant sense, can seem, at first blush, to be the kind of formal assimilation characteristic of natural bodies acting upon one another more generally. This, however, is misleading in at least two respects. 

First, as we shall see, Priscian will insist that the sense-organ is not a mere natural body but that life inheres in it. It is animated by the perceptive soul. Being animated makes a difference to how exactly it may be affected. The life that inheres in the eye contributes to the way in which it may be affected from without.

% A dead eye may be affected from without just as much as a living eye of a whole and healthy human perceiver. The interior of the dead eye may be illuminated, say, but it won't be affected in the same way as the living eye. It won't be subject, for example, to a pattern of retinal stimulation. 

Second, the formal assimilation involved in perception is not confined to the way in which the sense-organ assimilates to the form of the perceived object that acts upon it. Priscian makes this clear in the final line of the quoted passage. Here we have an allusion to Aristotle's definition of perception as the assimilation of form without matter (\emph{De anima} 2.5 418\( ^{a} \)3--6, 2.12 424\( ^{a} \)18--23), though as Huby observes, it is unclear whether the ``he himself'' is meant ot refer to Aristotle or to Theophrastus restating the Aristotelian doctrine \citep[49--50 n11]{Sorabji:1997ly}. Whereas Aristotle is willing to speak of the sense-organ (\emph{aisthêtêrion}) as assimilating to the form of the perceived object, at \emph{De anima} 2.5 418\( ^{a} \)3--6, 2.12 424\( ^{a} \)18--23 Aristotle is characterizing perception (\emph{aisthêsis}) as a kind of formal assimilation. While Priscian will discuss the formal assimilation involved in the object of perception acting upon the perceiver's sense-organ, his main focus will be on the formal assimilation involved in perception.

The objects of perception have ``forms and \emph{logoi}''. As these forms are assimilated in perception, we may confidently assume that they are understood to be sensible forms. This assumption is confirmed by Priscian using whiteness as an example of such a form (\emph{Metaphrasis} 3.3). Priscian's talk of the \emph{logoi} of perceptible objects is arguably an interpretation of an occurence of \emph{logos} in \emph{De anima} 424\( ^{a} \)25. Subsequent occurrences of \emph{logos} in the \emph{De anima} passage refer not to the objects of perception but to the perceptual capacity, such as at \emph{De anima} 424\( ^{a} \)28 where Aristotle discusses the \emph{logos} of the sense---here standardly translated as ``ratio''---being destroyed by overly strong stimulation. And Priscian himself will speak of \emph{logos} in connection with the substance of the perceptive soul, thus following, in his own manner, this second class of occurrences of \emph{logos} in the \emph{De anima} discussion. 

So perception is meant to assimilate to the form (and \emph{logos}?) of its object. In what does this assimilation consist in, asks Theophrastus. For it is absurd to suppose that the sense-organ becomes white when viewing a white thing (though, notoriously, some commentators attribute such a view to Aristotle \citealt{Slakey:1961ss, Sorabji:1974fk,Everson:1997ep}, implausibly to my mind \citealt{Kalderon:2015fr}). And it is even more absurd to suppose that the soul becomes white when seeing a white thing (though this conclusion was embraced by William Crathorn in his commentary on Lombard's \emph{Sentences}: ``A soul seeing and intellectively cognizing color is truly colored,'' \emph{Quaestiones super librum sententiarum} q. 1 concl. 7 \citealt[288]{Pasnau:2002pb}, prompting Robert Holcot to compare the soul, as Crathorn conceived of it, to a chameleon; see \citealt[chapter 1.1]{Pasnau:1997aa} for discussion). But if neither the sense organ nor the perceptive soul become like, in the most straightforward sense, the perceived object actually is, then in what does perception's formal assimilation consist in? That is the \emph{aporia} posed by Theophratus that occasions Prician's account of the process of formal assimilation in perception.

The \emph{logoi} associated with the soul (as opposed to the \emph{logoi} associated with the objects of perception---how exactly they are related is unclear) are epistemically significant. The awareness of the sensible and corporeal afforded by perception is conceived to be a mode of knowledge. All modes of knowledge involve gathering together the object of knowledge into an indivisible encompassment (\emph{Metaphrasis} 1.11--13). 

Perception, as Priscian conceives of it, is a mode of recognition. It is a distinctively perceptual mode of recognition, albeit as conceived by a Platonist. Specifically, perception involves the judgment (\emph{krisis}) and understanding (\emph{sunesis}) of the perceptive soul (\emph{Metaphrasis} 7.15--16). Of course, \emph{krisis} can mean ``discrimination'' in a sense that need not imply judgment, but, among the late Platonists, in discussions of perception, it typically has the more cognitively loaded sense of ``judgment''. But again, this is the judgment of the perceptive, as opposed to the rational, soul. This judgment involves the perceptive soul's projection of a \emph{logos} native to its substance or essence and fitting it to appearance of the form of the perceived object. 


% section theophrastus_emph_aporia (end)

\section{Perception's Formal Assimilation} % (fold)
\label{sec:perception_s_formal_assimilation}

On Priscian's account, the soul is active in three moments in the process of perception's formal assimilation to its object. Only in the third moment is perception perfected and the soul is made like that object. 

\emph{First moment}. 

\emph{Second moment}.

\emph{Third moment}.

% section perception_s_formal_assimilation (end)

\section{Some Related Problems \ldots} % (fold)
\label{sec:some_related_problems_ldots}

% section some_related_problems_ldots (end)

\section{\ldots and Priscian's Solution} % (fold)
\label{sec:_ldots_and_priscian_s_solution}

% section _ldots_and_priscian_s_solution (end)

% No cite

\bibliographystyle{plainnat}
\bibliography{Philosophy}

\end{document}
