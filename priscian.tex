%!TEX TS-program = xelatex 
%!TEX TS-options = -synctex=1 -output-driver="xdvipdfmx -q -E"
%!TEX encoding = UTF-8 Unicode
%
%  priscian
%
%  Created by Mark Eli Kalderon on 2016-07-26.
%  Copyright (c) 2016. All rights reserved.
%

\documentclass[12pt]{article} 

% Definitions
\newcommand\mykeywords{Priscian, perception}
\newcommand\myauthor{Mark Eli Kalderon}
\newcommand\mytitle{Priscian on Perception}

% Packages
\usepackage{geometry} \geometry{a4paper} 
\usepackage{url}
% \usepackage{txfonts}
\usepackage{color}
\usepackage{enumerate}
\definecolor{gray}{rgb}{0.459,0.438,0.471}
% \usepackage{setspace}
% \doublespace % Uncomment for doublespacing if necessary
% \usepackage{epigraph} % optional

% XeTeX
\usepackage[cm-default]{fontspec}
\usepackage{xltxtra,xunicode}
\defaultfontfeatures{Scale=MatchLowercase,Mapping=tex-text}
\setmainfont{Hoefler Text}

% Bibliography
\usepackage[round]{natbib}

% Title Information
\title{\mytitle}
\author{\myauthor} 
\date{} % Leave blank for no date, comment out for most recent date

% PDF Stuff
\usepackage[plainpages=false, pdfpagelabels, bookmarksnumbered, backref, pdftitle={\mytitle}, pdfauthor={\myauthor}, pdfkeywords={\mykeywords}, xetex, colorlinks=true, citecolor=gray, linkcolor=gray, urlcolor=gray, unicode=true]{hyperref} 

%%% BEGIN DOCUMENT
\begin{document}

% Title Page
\maketitle
% \begin{abstract} % optional
% \noindent
% \end{abstract}
% \vskip 2em \hrule height 0.4pt \vskip 2em
% \epigraph{} % optional; make sure to uncomment \usepackage{epigraph}

% Layout Settings
\setlength{\parindent}{1em}

% Main Content

\section{On the Significance of Priscian's account} % (fold)
\label{sec:on_the_significance_of_priscian_s_account}

Priscian of Lydia's \emph{Metaphrasis in Theophrastum} has been a rich doxographical source for scholars interested in Theophrastus and Iamblichus. The \emph{Metaphrasis} is billed as a paraphrase of the fifth book of the now lost work \emph{Physica} by Aristotle's student and successor Theophrastus. We know from Themistius that the fourth and fifth books of \emph{Physica} concerned the soul. And from the \emph{Metaphrasis}, we know that it consisted, at least in part, in Theophrastus raising some questions concerning Aristotle's \emph{De anima}. In the \emph{Metaphrasis}, Priscian endeavours to answer these questions relying upon the psychological doctrines of Iamblichus. Priscian answers Theophrastus' questions \emph{in propria voce}. Priscian is principally concerned to set down the truth of the matter, as he understands it, rather than using Theophrastus' questions as an opportunity to engage in a closer exegesis of \emph{De anima}. What Aristotle might have meant matters little, especially if it is potentially at variance with the truth of the matter as revealed by Iamblichus. (On Priscian's method as compared to Pseudo-Simplicius, see \citealt[7--10]{Steel:1978th}.)

The \emph{Metaphrasis}, in the fragmentary state that it has come down to us, begins with a puzzle or \emph{aporia} raised by Theophrastus concerning the formal assimilation involved in perception. If perception involves somehow becoming like the perceived object, then what does this becoming like consist in? ``For with sense-organs, and even more with the soul, the capacity to become like <an object> in color and tastes and sound and shape seems absurd'' (Priscian, \emph{Metaphrasis} 1 5; Pamela Huby in \citealt{Sorabji:1997ly}). This \emph{aporia} occasions Priscian's account of the process involved in perception's formal assimilation to its object. In this ``inaugural'' discussion of the \emph{Metaphrasis}, Priscian is principally concerned to establish that the perceptive soul is the efficient cause of its formal assimilation to the object of perception (\emph{Metaphrasis} 1.13--16, 3.7--8).

Priscian's account thus belongs to a general class of such accounts where the soul is the efficient cause of its likeness to the perceived object. Such accounts can be found among late Platonists such as Priscian and Pseudo-Simiplicius (possibly one and the same, see \citealt{Bossier:1972rp}, \citealt{Steel:1978th} Steel's introduction to his translation of \emph{In de anima} in \citealt[103--140]{Sorabji:1997ly}), Christian Platonists such as Augustine and the medieval thinkers that took inspiration from him, and among the thinkers involved in the Renaissance revival of Simplician Averroism (for discussion see \citealt[chapter 8]{Spruit:1995fh}; Jacopo Zabarella offers an interesting argument from selective attention against the Simplicianism of his teacher Marcantonnio Genua in \emph{Liber de densu agente}). All such accounts face a general problem given that the sensible form of the perceived object is excluded as explanatorily relevant to the perceptive soul's formal assimilation, being confined to at best occasioning the soul's activity. This explanatory exclusion is the basis of related problems raised by, \emph{inter alia}, Plotinus in the opening \emph{aporia} of Ennead 3.6, in Aquinas' criticism of Augustine in \emph{Quaestiones disputatae de veritate} 10.6, and Duns Scotus' criticism of Olivi in his \emph{Ordinatio} 1.3.3.4. The principle interest of Priscian's account is that it, along with Pseudo-Simiplicius' account in \emph{In de anima}, represents a way out of this general difficulty, though one that Aquinas judged absurd.

% section on_the_significance_of_priscian_s_account (end)

\section{Theophrastus' \emph{Aporia}} % (fold)
\label{sec:theophrastus_emph_aporia}

% section theophrastus_emph_aporia (end)

\section{Perception's Formal Assimilation} % (fold)
\label{sec:perception_s_formal_assimilation}

% section perception_s_formal_assimilation (end)

\section{Some Related Problems \ldots} % (fold)
\label{sec:some_related_problems_ldots}

% section some_related_problems_ldots (end)

\section{\ldots and Priscian's Solution} % (fold)
\label{sec:_ldots_and_priscian_s_solution}

% section _ldots_and_priscian_s_solution (end)

% No cite

\bibliographystyle{plainnat}
\bibliography{Philosophy}

\end{document}
